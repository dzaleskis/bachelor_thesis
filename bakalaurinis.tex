\documentclass{VUMIFInfBakalaurinis}
\usepackage{algorithmicx}
\usepackage{algorithm}
\usepackage{algpseudocode}
\usepackage{amsfonts}
\usepackage{amsmath}
\usepackage{bm}
\usepackage{caption}
\usepackage{color}
\usepackage{float}
\usepackage{graphicx}
% \usepackage{hyperref}  % Nuorodų aktyvavimas
\usepackage{listings}
\usepackage{subfig}
\usepackage{url}
\usepackage{wrapfig}


% Titulinio aprašas
\university{Vilniaus universitetas}
\faculty{Matematikos ir informatikos fakultetas}
\institute{Informatikos institutas}  % Užkomentavus šią eilutę - institutas neįtraukiamas į titulinį
\department{Informatikos katedra}
\papertype{Baigiamasis bakalauro darbas}
\title{Rikiavimo tobulinimas genetiniais algoritmais}
\titleineng{Improving Sorting with Genetic Algorithms}
\status{4 kurso 2 grupės studentas}
\author{Deividas Zaleskis}
\supervisor{Irmantas Radavičius}
\reviewer{doc. dr. Vardauskas Pavardauskas}
\date{Vilnius \\ \the\year}

% Nustatymai
% \setmainfont{Palemonas}   % Pakeisti teksto šriftą į Palemonas (turi būti įdiegtas sistemoje)
\bibliography{bibliografija} 

\begin{document}
\maketitle

\tableofcontents

\sectionnonum{Sąvokų apibrėžimai}
Sutartinių ženklų, simbolių, vienetų ir terminų sutrumpinimų sąrašas (jeigu
ženklų, simbolių, vienetų ir terminų bendras skaičius didesnis nei 10 ir
kiekvienas iš jų tekste kartojasi daugiau nei 3 kartus).

\sectionnonum{Įvadas}
Įvade apibūdinamas darbo tikslas, temos aktualumas ir siekiami rezultatai.

\section{Pagrindinė tiriamoji dalis}
Pagrindinėje tiriamojoje dalyje aptariama ir pagrindžiama tyrimo metodika; pagal
atitinkamas darbo dalis, nuosekliai, panaudojant lyginamosios analizės, klasifikacijos,
sisteminimo metodus bei apibendrinimus, dėstoma sukaupta ir išanalizuota medžiaga. 

\sectionnonum{Išvados}
Išvadose ir pasiūlymuose, nekartojant atskirų dalių apibendrinimų,
suformuluojamos svarbiausios darbo išvados, rekomendacijos bei pasiūlymai.

\sectionnonum{Conclusions}
Šiame skyriuje pateikiamos išvados (reziume) anglų kalba.


\printbibliography[heading=bibintoc]

\appendix  % Priedai


\end{document}
