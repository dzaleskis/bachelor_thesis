\documentclass{VUMIFInfBakalaurinis}
\usepackage{algorithmicx}
\usepackage{algorithm}
\usepackage{algpseudocode}
\usepackage{amsfonts}
\usepackage{amsmath}
\usepackage{bm}
\usepackage{caption}
\usepackage{color}
\usepackage{float}
\usepackage{graphicx}
% \usepackage{hyperref}  % Nuorodų aktyvavimas
\usepackage{listings}
\usepackage{subfig}
\usepackage{url}
\usepackage{wrapfig}


% Titulinio aprašas
\university{Vilniaus universitetas}
\faculty{Matematikos ir informatikos fakultetas}
\institute{Informatikos institutas}
\department{Informatikos katedra}
\papertype{Baigiamasis bakalauro darbas}
\title{Rikiavimo tobulinimas genetiniais algoritmais}
\titleineng{Improving Sorting with Genetic Algorithms}
\status{4 kurso 2 grupės studentas}
\author{Deividas Zaleskis}
\supervisor{Irmantas Radavičius}
\reviewer{doc. dr. Vardauskas Pavardauskas}
\date{Vilnius \\ \the\year}

\bibliography{bibliografija} 

\begin{document}
\maketitle

\tableofcontents

% ar verta tureti santrauka?

\sectionnonum{Sąvokų apibrėžimai}
Sutartinių ženklų, simbolių, vienetų ir terminų sutrumpinimų sąrašas (jeigu
ženklų, simbolių, vienetų ir terminų bendras skaičius didesnis nei 10 ir
kiekvienas iš jų tekste kartojasi daugiau nei 3 kartus).

\sectionnonum{Įvadas}
% Įvade apibūdinamas darbo tikslas, temos aktualumas ir siekiami rezultatai.

Duomenų rikiavimas yra vienas aktyviausiai tiriamų uždavinių informatikos moksle.
Iš dalies tai lemia rikiavimo uždavinio prieinamumas ir sąlygos paprastumas.
Formaliai rikiavimo uždavinys formuluojamas taip:
duotai baigtinei palyginamų elementų sekai $S = (s_1, s_2, ..., s_n)$ pateikti tokį
kėlinį, kad duotosios sekos elementai būtų išdėstyti monotonine (didėjančia arba mažėjančia) tvarka.
Kadangi rikiavimo uždavinio sąlyga yra paprasta, tai suteikia didelę galimų implementacijų įvairovę.
Todėl nauji rikiavimo algoritmai ir įvairūs patobulinimai egzistuojantiems algoritmams yra kuriami ir dabar.

Rikiavimo uždavinys yra fundamentalus, kadangi efektyvus rikiavimas padeda pagrindą
efektyviam kitų uždavinių sprendimui.
Kaip to pavyzdį galima pateikti dvejetainės paieškos algoritmą, kurio prielaida,
jog duomenys yra išrikiuoti, leidžia sumažinti paieškos laiko sudėtingumą iki $O(log\,n)$.
Efektyvus rikiavimas taip pat svarbus duomenų normalizavimui bei pateikimui žmonėms lengvai suprantamu formatu.
Kadangi duomenų rikiavimas yra fundamentalus uždavinys, net ir nežymūs patobulinimai žvelgiant bendrai
gali atnešti didelę naudą.

Rikiavimo uždaviniui spręsti egzistuoja labai įvairių algoritmų.
Plačiausiai žinomi yra klasikiniai rikiavimo algoritmai: rikiavimas sąlaja (angl. merge sort), rikiavimas įterpimu (angl. insertion sort) ir greitojo rikiavimo algoritmas (angl. quicksort).
Tiesa, šie algoritmai turi įvairių trukūmų:
rikiavimas sąlaja veikia lėčiau nei nestabilūs algoritmai, rikiavimas įterpimu yra efektyvus tik kai rikiuojamų duomenų dydis yra mažas, o
greitojo rikiavimo algoritmas netinkamai parinkus slenkstį gali degraduoti į $O(n^2)$ laiko sudėtingumą.
Todėl šiuo metu praktikoje plačiausiai naudojami yra hibridiniai rikiavimo algoritmai, kurie apjungia kelis klasikinius algoritmus į vieną panaudodami jų geriausias savybes.
Pavyzdžiui, C++ programavimo kalbos standartinėje bibliotekoje pateikiamas introsort algoritmas įprastai naudoja greitojo rikiavimo algoritmą,
pasiekus tam tikrą rekursijos gylį yra naudojamas rikiavimas krūva (angl. heapsort) siekiant išvengti degradacijos į $O(n^2)$ laiko sudėtingumą,
o kai rikiuojamų duomenų dydis yra pakankamai mažas, pasitelkiamas rikiavimas įterpimu, kadangi su mažais duomenų dydžiais jis yra efektyvesnis.
Apibendrinant, pasitelkiant įvairius rikiavimo algoritmus ir jų unikalias savybes yra įmanoma rikiavimo uždavinį spręsti efektyviau.



\section{Pagrindinė tiriamoji dalis}
Pagrindinėje tiriamojoje dalyje aptariama ir pagrindžiama tyrimo metodika; pagal
atitinkamas darbo dalis, nuosekliai, panaudojant lyginamosios analizės, klasifikacijos,
sisteminimo metodus bei apibendrinimus, dėstoma sukaupta ir išanalizuota medžiaga. 

\sectionnonum{Išvados}
Išvadose ir pasiūlymuose, nekartojant atskirų dalių apibendrinimų,
suformuluojamos svarbiausios darbo išvados, rekomendacijos bei pasiūlymai.

\sectionnonum{Conclusions}
Šiame skyriuje pateikiamos išvados (reziume) anglų kalba.


\printbibliography[heading=bibintoc]

\appendix  % Priedai


\end{document}
